\section{Schematized model of the Waal River}
\newrefsegment

%---------------------------------------------------------------------------------
\section*{Purpose}
In this exercise, we consider a 90 km long stretch of the Waal River. For simplicity, the stretch is represented as a straight flume, with constant bed slope. Depending on the flow discharge, the bed roughness and the downstream water level boundary condition, different flow regimes can occur. The simplest form - which can also be computed analytically - is a uniform channel flow, where a constant (equilibruim) water depth $H^{eq}$ and a constant flow velocity $u$ develop along the full length of the model.
A different flow situation can occur when the discharge is higher than the equilibrium discharge (belonging to the equilibrium depth $H^{eq}$, e.g. in the case of a river flood wave. 

The aim of this exercise is twofold. First, we inspect whether a 1D numerical model (with many assumptions!) can represent the analytical solution of uniform channel flow. Second, we increase the discharge to model a flood wave. For this situation one can compute and inspect the travel/arrival times of such a flood wave.

%---------------------------------------------------------------------------------
\section*{Approach}
First we set up the model to compute uniform channel flow. Our river stretch has the following dimensions:

\begin{itemize}
  \item Length $L$ = 90 km
  \item Width $W$ = 250 m
  \item bed slope $i_b$ = 1e-4 m/m (positive downwards)
\end{itemize}

Additionally, we prescribe a fixed discharge $Q$ = 1000 m$^3$/s and a Ch\'ezy value of 50 m$^{1/2}$/s. Using the Ch\'ezy formula we can now compute the equilibrium depth that should theoretically develop for these channel/flow characteristics:

\begin{eqnarray}
	u &=& C \sqrt{R i_b} \,, \qquad \textrm{or} \nonumber \\
	u &=& C \sqrt{H^{eq} i_b} \,.
\end{eqnarray}

where we assumed $R = H$, which can be consider valid for our wide channel/river section. Now we use $u = Q / (BH)$ and rewrite the equation for the equilibrium depth $H^{eq}$:

\begin{eqnarray}
	H^{eq} = \left( \frac{Q}{B C \sqrt{i_b}} \right)^{2/3} \,.
\end{eqnarray}

Filling in the chose values for $Q$, $B$, $C$ and $i_b$, one obtains an equilibrium depth $H^{eq}$ = 4 m.
This flow situation is depicted in Figure~\ref{fig:uniform_channel_flow}.

\begin{figure}[H]
    \centering
    \includegraphics*[width=\textwidth]{figures/uniform_channel_flow.png}
    \caption{The uniform channel flow situation for the schematic model of the Waal River.}
		\label{fig:uniform_channel_flow}
\end{figure}
\ldots

%---------------------------------------------------------------------------------
\section*{Conclusion} 
\ldots

%---------------------------------------------------------------------------------
\section*{Model description}
\ldots

%---------------------------------------------------------------------------------
\section*{Results}
\ldots

%---------------------------------------------------------------------------------
\section*{Analysis of results}
\ldots

%---------------------------------------------------------------------------------
\printrefsegment
